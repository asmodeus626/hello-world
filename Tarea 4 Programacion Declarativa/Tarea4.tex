\documentclass{article}
\usepackage[T1]{fontenc}
\usepackage[utf8]{inputenc}
\usepackage[spanish]{babel}
\usepackage{amssymb, amsmath, amsbsy} % simbolitos
\usepackage{upgreek} % para poner letras griegas sin cursiva
\usepackage{cancel} % para tachar
\usepackage{mathdots} % para el comando \iddots
\usepackage{mathrsfs} % para formato de letra
\usepackage{stackrel} % para el comando \stackbin
\usepackage{dsfont} % para conjuntos de números.
\usepackage{multirow} % para tablas
\usepackage{synttree}
\usepackage[all]{xy}
\title{Tarea 4 \\Programación declarativa}
\author{Peto Gutierrez Emmanuel \\ Ernesto Rubén Palacios Gómez}
\begin {document}
\maketitle

1.- Por definición $a \leq b$ implica que $\exists c \in  \mathbb{N}$ tal que $a+c=b$.\\

Voy a etiquetar las flechas de manera única con una terna, de forma que a la flecha $a \xrightarrow{f} b$ le corresponde la terna $f=(a,b,c)$. Por ejemplo $id_{a}=(a,a,0)$ o si $b=s(a)$ entonces a la flecha $a \rightarrow b$ le corresponde $(a,b,1)$. La composición se definirá así: Si tenemos que $a \leq b$ y $b \leq d$ y tenemos que $a+c=b$ y $b+e=d$, entonces $(b,d,e) \circ (a,b,c) = (a,d,e+c)$. La primera entrada de la terna de la izquierda debe ser igual a la segunda entrada de la terna de la derecha.\\

Veremos que se cumple la asociatividad y la identidad por la izquierda y derecha.\\

\textbf{Asociatividad}: Sean $a,b,c,d \in \mathbb{N}$ tal que $a \leq b \leq c \leq d$. Sean $f,g,h$ flechas tal que $a \xrightarrow{f} b, b \xrightarrow{g} c, c \xrightarrow{h} d$, donde $f=(a,b,x), g=(b,c,y), h=(c,d,z)$.\\ $g \circ f = (a,c,x+y)$ \\ $h \circ g = (b,d,y+z)$ \\
Luego, $h \circ (g \circ f) = (a,d,x+y+z) = (h \circ g) \circ f \blacksquare$ \\

\textbf{Identidad izquierda}: $id_{b} \circ f$ donde $a \xrightarrow{f} b.$ \\ $(b,b,0) \circ (a,b,c)=(a,b,c+0)=(a,b,c) = f \blacksquare$ \\

\textbf{Identidad derecha}: $f \circ id_{a} = (a,b,c) \circ (a,a,0)=(a,b,c+0)=(a,b,c)=f \blacksquare$ \\

Un funtor $F: \mathbb{N} \rightarrow \mathbb{N}$ podría ser la función sucesor definida así: \\
$F(a)=S(a)$ \\
$F(a,b,c)=(S(a),S(b),c)$ \\ \ \\

2.- Sean $f:x \rightarrow y, g:y \rightarrow z$ funciones de Haskell y $x,y,z$ tipos de Haskell. \\ $g \circ f:x \rightarrow z$. \\ $F(x)=a \rightarrow x$ \\ $F(y)=a \rightarrow y$ \\ $F(z)=a \rightarrow z$    \\ $F(f) = F(x) \rightarrow F(y) = (a \rightarrow x) \rightarrow (a \rightarrow y)$ \\ $F(g) = F(y) \rightarrow F(z) = (a \rightarrow y) \rightarrow (a \rightarrow z)$ \\ $ F(g \circ f) = (a \rightarrow x)     \rightarrow (a \rightarrow z) = F(g) \circ F(f) \blacksquare$ Vemos que preserva la composición.\\

Sea $id_{x} = x \rightarrow x$ donde x es un tipo.\\
$F(x)=a \rightarrow x$\\
$F(id_{x})= (a \rightarrow x) \rightarrow (a \rightarrow x) = id_{F(x)} \blacksquare$ Preserva la identidad. \\

3.- Sean $f,g,id_{x}$ definidos como en (2).\\
$F_{1} (x)=(a,b) \rightarrow x$\\
$F_{1} (y)=(a,b) \rightarrow y$\\
$F_{1} (z)=(a,b) \rightarrow z$\\
$F_{1} (f)=((a,b) \rightarrow x) \rightarrow ((a,b) \rightarrow y)$\\
$F_{1} (g)=((a,b) \rightarrow y) \rightarrow ((a,b) \rightarrow z)$\\
$F_{1} (g \circ f)=((a,b) \rightarrow x) \rightarrow ((a,b) \rightarrow z)$\\
$F_{1}(g) \circ F_{1}(f)=((a,b) \rightarrow x) \rightarrow ((a,b) \rightarrow z)=F_{1} (g \circ f) \blacksquare$ Preserva la composición\\
$F_{1} (id_{x}) = ((a,b) \rightarrow x) \rightarrow ((a,b) \rightarrow x)=id_{F_{1}(x)} \blacksquare$ Preserva identidades.\\

$F_{2} (f) = (a \rightarrow (b \rightarrow x)) \rightarrow (a \rightarrow (b \rightarrow y))$\\
$F_{2} (g) = (a \rightarrow (b \rightarrow y)) \rightarrow (a \rightarrow (b \rightarrow z))$\\
$F_{2} (g \circ f) = (a \rightarrow (b \rightarrow x)) \rightarrow (a \rightarrow (b \rightarrow z))$\\
$F_{2} (g) \circ F_{2} (f) = (a \rightarrow (b \rightarrow x)) \rightarrow (a \rightarrow (b \rightarrow z)) \blacksquare$ Preserva la composición.\\
$F_{2} (id_{x}) = (a \rightarrow (b \rightarrow x)) \rightarrow (a \rightarrow (b \rightarrow x)) = id_{F_{2}(x)} \blacksquare$ Preserva la identidad.\\

4.- \\

\[
\xymatrix{
F_{1}(a) \ar[d]^{\eta_{1a}} \ar[r]^{F_{1}(f)} & F_{1}(b) \ar[d]^{\eta_{1b}} \\
F_{2}(a) \ar[r]^{F_{2}(f)} \ar[u]^{\eta_{2a}} & F_{2}(b) \ar[u]^{\eta_{2b}}
}
\]

Las composiciones $\eta_{1} \circ \eta_{2}$ y $\eta_{2} \circ \eta_{1}$ son funciones identidad, entonces solo se cumple la propiedad $\eta_{1} \circ \eta_{2} = F_{2}$ y $\eta_{2} \circ \eta_{1} = F_{1}$ si $F_{1}$ y $F_{2}$ son funtores identidad.\\

5.-\\ $\eta : [] \Rightarrow []$ \\ $\eta_{a} : [a] \rightarrow [a]$ \\ $xs \mapsto sort \; xs$\\

$\eta$ no es una transformación natural. Esto se debe a que sort solo se puede aplicar a listas cuyos elementos sean comparables, es decir, que tengan definidas las operaciones $<,>,=$. Supongamos que tenemos una función de Haskell $f:Int \rightarrow Bool$ que decide si un número es par. Supongamos que $\ell$ es una lista de tipo $Int$, la operación $map \; f \; (sort \; \ell)$ está definida, pero $sort \; (map \; f \; \ell)$ no está definida, porque no podemos ordenar tipo $Bool$. Entonces $(map \; f) \; (sort \; \ell) \neq sort \; ((map \; f) \; \ell)$. \\

6.-\\
Definamos la función $Maybe\_fun$ así:\\
$Maybe\_fun :: (a \rightarrow b) \rightarrow Maybe \; a$\\
$Maybe\_fun \; f \; Nothing = Nothing$\\
$Maybe\_fun \; f \; (Just \; a) = Just \; (f \; a)$\\

Utilizaré el término $f(x)$ para describir el resultado de aplicar $f$ a $x$ en vez de $f \; x$, como sería en Haskell. Sea $f:a \rightarrow b$, veremos que el siguiente diagrama conmuta:

\[
\xymatrix{
Maybe \; a \ar[d]^{\eta_{a}} \ar[r]^{Maybe\_fun f} & Maybe \; b \ar[d]^{\eta_{b}} \\
[a] \ar[r]^{map \; f} & [b]
}
\]
\\

Caso $Nothing$:\\
$map\;f\;(\eta_{a}\;Nothing)=map\;f\;[]=[]$ \\
$\eta_{b}\;(Maybe\_fun\;f\;Nothing)= \eta_{b}\;Nothing=[]$ \\
Por lo tanto $map\;f\;(\eta_{a}\;Nothing)=\eta_{b}\;(Maybe\_fun\;f\;Nothing) \blacksquare$\\

Caso $x$:\\
$map\;f\;(\eta_{a}\;(Just\;x))=map\;f\;[x]=[f(x)]$\\
$\eta_{b}\;(Maybe\_fun\;f\;(Just\;x))=\eta_{b}\;(Just\;f(x))=[f(x)]$\\
Por lo tanto $map\;f\;(\eta_{a}\;(Just\;x))=\eta_{b}\;(Maybe\_fun\;f\;(Just\;x)) \blacksquare$
\end {document}